\documentclass[11pt,a4paper]{article}
\usepackage[utf8]{inputenc}
\usepackage[T1]{fontenc}
\usepackage{geometry}
\usepackage{fancyhdr}
\usepackage{graphicx}
\usepackage{xcolor}
\usepackage{hyperref}
\usepackage{array}
\usepackage{longtable}
\usepackage{booktabs}
\usepackage{enumitem}
\usepackage{amsmath}
\usepackage{amsfonts}
\usepackage{amssymb}

% Page setup
\geometry{left=2.5cm,right=2.5cm,top=3cm,bottom=3cm}
\pagestyle{fancy}
\fancyhf{}
\fancyhead[L]{KGEN Project Charter}
\fancyhead[R]{\thepage}
\fancyfoot[C]{Generated by KGEN Self-Hosting System}

% Colors
\definecolor{kgenblue}{RGB}{0,102,204}
\definecolor{kgengray}{RGB}{85,85,85}

% Hyperref setup
\hypersetup{
    colorlinks=true,
    linkcolor=kgenblue,
    filecolor=kgenblue,
    urlcolor=kgenblue,
    citecolor=kgenblue
}

\begin{document}

% Title Page
\begin{titlepage}
    \centering
    {\Huge\bfseries KGEN v1 Project Charter\par}
    \vspace{1.5cm}
    {\Large\itshape KGEN - Knowledge Graph Engineering Next Generation\par}
    \vspace{1cm}
    {\large Version 1.0.0\par}
    \vspace{0.5cm}
    {\large Generated: 2025-09-12\par}
    \vspace{2cm}
    
    \begin{center}
    \begin{tabular}{|l|l|}
    \hline
    \textbf{Project Status} & {{ project.status | title }} \\
    \hline
    \textbf{Charter Version} & {{ charter.version | default('1.0') }} \\
    \hline
    \textbf{Created} & 2024-01-01 \\
    \hline
    \textbf{Last Modified} & 2025-01-12 \\
    \hline
    \end{tabular}
    \end{center}
    
    \vfill
    {\large\textcolor{kgengray}{Generated by KGEN Self-Hosting Bootstrap System}\par}
\end{titlepage}

% Table of Contents
\tableofcontents
\newpage

% Executive Summary
\section{Executive Summary}

Comprehensive charter for KGEN development and enterprise adoption

This charter defines the scope, objectives, stakeholders, and success criteria for the KGEN - Knowledge Graph Engineering Next Generation project. The project leverages revolutionary RDF-based code generation technology to transform enterprise software development.

\subsection{Key Benefits}
\begin{itemize}[leftmargin=2cm]
    \item Revolutionary code generation using RDF graphs and SHACL validation
    \item Self-hosting capabilities with complete bootstrap demonstration
    \item Enterprise-grade security, scalability, and reliability
    \item Comprehensive testing and validation framework
    \item Automated documentation and workshop material generation
\end{itemize}

% Project Overview
\section{Project Overview}

\subsection{Project Description}
Revolutionary code generation system using RDF graphs and SHACL validation

\subsection{Project Scope}
The KGEN project encompasses the following key areas:
\begin{itemize}[leftmargin=2cm]
    \item Core RDF processing and SHACL validation engine
    \item Template-based code generation system
    \item Self-hosting bootstrap capabilities
    \item Enterprise integration and security features
    \item Comprehensive testing and validation framework
    \item Workshop materials and training content
\end{itemize}

% Stakeholders
\section{Stakeholders}


\subsection{{{ stakeholder.name }}}
\begin{tabular}{|p{3cm}|p{10cm}|}
\hline
\textbf{Role} & {{ stakeholder.role | replace('_', ' ') | title }} \\
\hline
\textbf{Responsibility} & {{ stakeholder.responsibility }} \\
\hline
\textbf{Interest} & {{ stakeholder.interest }} \\
\hline
\textbf{Influence} & {{ stakeholder.influence | title }} \\
\hline
\textbf{Communication} & {{ stakeholder.communication }} \\
\hline
\end{tabular}



% Critical To Quality (CTQ) Metrics
\section{Critical To Quality (CTQ) Metrics}

The following CTQ metrics define the measurable success criteria for the project:

\begin{longtable}{|p{3cm}|p{3cm}|p{3cm}|p{2cm}|p{2cm}|}
\hline
\textbf{Metric} & \textbf{Target} & \textbf{Measure} & \textbf{Threshold} & \textbf{Priority} \\
\hline
\endfirsthead
\hline
\textbf{Metric} & \textbf{Target} & \textbf{Measure} & \textbf{Threshold} & \textbf{Priority} \\
\hline
\endhead

{{ ctq.title }} & {{ ctq.target }} & {{ ctq.measure }} & {{ ctq.threshold }} & {{ ctq.priority | title }} \\
\hline

\end{longtable}

% Project Milestones
\section{Project Milestones}

\begin{longtable}{|p{0.8cm}|p{4cm}|p{2.5cm}|p{2cm}|p{4.5cm}|}
\hline
\textbf{ID} & \textbf{Milestone} & \textbf{Due Date} & \textbf{Status} & \textbf{Deliverables} \\
\hline
\endfirsthead
\hline
\textbf{ID} & \textbf{Milestone} & \textbf{Due Date} & \textbf{Status} & \textbf{Deliverables} \\
\hline
\endhead

{{ loop.index }} & {{ milestone.title }} & {{ milestone.dueDate | date('YYYY-MM-DD') }} & {{ milestone.status | replace('_', ' ') | title }} & {{ milestone.deliverables }} \\
\hline

\end{longtable}

% Technical Architecture
\section{Technical Architecture}

\subsection{Architecture Overview}
The KGEN system follows a microservices architecture pattern with containerized deployment strategy. The system is designed for horizontal scalability.

\subsection{System Components}

\subsubsection{{{ component.title }}}
\begin{tabular}{|p{3cm}|p{10cm}|}
\hline
\textbf{Responsibility} & {{ component.responsibility }} \\
\hline
\textbf{Technology} & {{ component.technology }} \\
\hline
\textbf{Interface} & {{ component.interface }} \\
\hline
\end{tabular}



% Self-Hosting Capabilities
\section{Self-Hosting Bootstrap System}

\subsection{Overview}
KGEN Self-Hosting Bootstrap demonstrates KGEN's capability to generate its own documentation and validate its outputs through a complete dogfooding approach.

\subsection{Generated Artifacts}

\subsubsection{{{ artifact.title }}}
\begin{itemize}[leftmargin=2cm]
    \item \textbf{Source RDF:} {{ artifact.sourceRDF }}
    \item \textbf{Template:} {{ artifact.template }}
    \item \textbf{Output Formats:} {{ artifact.outputFormat | join(', ') }}
\end{itemize}


% Testing and Validation
\section{Testing and Validation}

\subsection{Test Suite Overview}
The validation suite provides 100\% coverage with full automation.

\subsection{Test Cases}

\subsubsection{{{ test.title }}}
\begin{tabular}{|p{3cm}|p{10cm}|}
\hline
\textbf{Verifies} & {{ test.verifies }} \\
\hline
\textbf{Criteria} & {{ test.criteria }} \\
\hline
\end{tabular}



% Risk Management
\section{Risk Management}

\subsection{Technical Risks}
\begin{itemize}[leftmargin=2cm]
    \item \textbf{RDF Complexity:} Mitigation through comprehensive SHACL validation
    \item \textbf{Performance:} Mitigation through optimized query engines and caching
    \item \textbf{Integration:} Mitigation through extensive testing and phased rollout
\end{itemize}

\subsection{Business Risks}
\begin{itemize}[leftmargin=2cm]
    \item \textbf{Adoption:} Mitigation through comprehensive training and support
    \item \textbf{Scalability:} Mitigation through horizontal scaling architecture
    \item \textbf{Maintenance:} Mitigation through automated testing and self-validation
\end{itemize}

% Success Criteria
\section{Success Criteria}

The project will be considered successful when:
\begin{enumerate}[leftmargin=2cm]
    \item All CTQ metrics meet or exceed defined thresholds
    \item All project milestones are completed on schedule
    \item Self-hosting capabilities are fully demonstrated
    \item All stakeholders approve final deliverables
    \item Production deployment is operational and stable
\end{enumerate}

% Approval
\section{Charter Approval}

\begin{center}
\begin{tabular}{|p{4cm}|p{4cm}|p{4cm}|}
\hline
\textbf{Role} & \textbf{Name} & \textbf{Date} \\
\hline
Project Sponsor & \rule{3cm}{0.4pt} & \rule{3cm}{0.4pt} \\
\hline
Technical Lead & \rule{3cm}{0.4pt} & \rule{3cm}{0.4pt} \\
\hline
Quality Assurance & \rule{3cm}{0.4pt} & \rule{3cm}{0.4pt} \\
\hline
Business Analyst & \rule{3cm}{0.4pt} & \rule{3cm}{0.4pt} \\
\hline
\end{tabular}
\end{center}

% Appendices
\appendix

\section{Glossary}
\begin{itemize}[leftmargin=2cm]
    \item \textbf{CTQ:} Critical To Quality - measurable characteristics that are important to customers
    \item \textbf{RDF:} Resource Description Framework - W3C standard for data interchange
    \item \textbf{SHACL:} Shapes Constraint Language - W3C standard for validating RDF data
    \item \textbf{Self-Hosting:} System's ability to generate and validate its own artifacts
    \item \textbf{Dogfooding:} Practice of using one's own product internally
\end{itemize}

\section{References}
\begin{itemize}[leftmargin=2cm]
    \item W3C RDF 1.1 Specification
    \item W3C SHACL Specification
    \item Project Management Institute (PMI) Standards
    \item Enterprise Architecture Best Practices
\end{itemize}

\end{document}